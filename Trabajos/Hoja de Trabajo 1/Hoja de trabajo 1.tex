\documentclass[12pt]{extreport}


% Uncomment the following line to allow the usage of graphics (.jpg, .png, etc.)
\usepackage[pdftex]{graphicx}
% Allow the change of line spacing
\usepackage{setspace}
% Allow the usage of umlauts and other non-ASCII characters
\usepackage[utf8]{inputenc}
\usepackage[12pt]{moresize}
\usepackage[margin=2.5cm,left=4cm]{geometry}
\usepackage{fancyhdr}
\usepackage{hyperref}
\usepackage{mathptmx}
\pagestyle{fancy}
\fancyhead{}
\fancyfoot{}
\fancyfoot[R]{\thepage\ }
\renewcommand{\headrulewidth }{0pt}


\newcommand{\horrule}[1]{\rule{\linewidth}{#1}}

% Start the document
\begin{document}

% Create a new 1st level heading
\begin{titlepage}
\begin{center}
\horrule{1pt}\\
[0.5cm]
\huge{\textbf{Hoja de Trabajo No.1}}\\
\textsc{\large INFORMÁTICA 2}\\
\textsc{\large UNIVERSIDAD DEL ISTMO}\\
[0.02cm]
\horrule{1pt}\\
[12.5cm]
\end{center}
\begin{flushright}
\textsc{\large Pablo José Ponce B.}\\
Carné No. 2017-1381
\end{flushright}
\end{titlepage}


\begin{center}
\section*{Que Haceres}\label{sec:quehaceres}
\end{center}
Una lista de que haceres con cosas especificas que se tienen que hacer varias veces, como Lavar los platos, ordenar la mesa, comprar comida, etc.

En esta lista se va a mostrar todas los atributos que el objeto abstracto de una lista "que haceres" tiene.\\

\subsection*{Atributos de una lista de que haceres}
\begin{description} 
\item[$\ast$ ]Tiene que tener una opción para crear los que haceres.
\subitem$\ast$ Con esta opción se agrega un ítem cualquiera para la lista de que haceres como "Lavar el carro". 
\item[$\ast$ ]Una opción para poder editar los ítems creados.
\subitem$\ast$ Si el usuario se equivoca o desea agregar más al título lo edita, así pasa del título original "Lavar el carro" a "Lavar los carros".
\item[$\ast$ ]Una forma para poder mover los ítems al lugar que desee el usuario.
\subitem$\ast$ Así si el usuario lo desea puede cambiar el orden de los ítems en la lista, del más importante al menos importante o como desee el usuario.
\item[$\ast$ ]Crear grupos para incorporar los ítems.
\subitem$\ast$ Digamos que el usuario llegó a tener una lista de 25 que haceres, pero no los quiere ordenar para no tener tantos. Los ordena por grupos "Casa", "Trabajo", "Universidad", etc.
\item[$\ast$ ]Poder eliminar los ítems creados.
\subitem$\ast$ En un momento el usuario ya no tiene carro, por consiguiente ya no tiene que lavar el carro, así que tiene que eliminar el ítem.
\end{description}
\newpage

\begin{center}
\section*{Que Hacer}\label{sec:quehacer}
\end{center}

Una lista de cosas que hacer, es diferente a las que haceres, en estas son cosas que se tienen que hacer una vez o tal vez más pero no de forma repetitiva, por ejemplo este trabajo es una cosa que hacer, ya que se tiene que entregar en un día.

Al igual que la sección anterior vamos a describir sus atributos (Algunos similares al anterior).\\

\subsection*{Atributos de una lista que hacer}
\begin{description}
\item[$\ast$ ]Creación de los ítems que se necesiten (Esto incluye la asignación de hora, fecha, alarma, etc.)
\subitem$\ast$ Se crea un nuevo ítem en lo que el usuario tiene que hacer en el día, se asigna el día de la alarma, hora y la alarma que se desea poner para este ítem.
\item[$\ast$ ]Incluir (si necesario) prioridades en las secciones creadas.
\subitem$\ast$ Dependiendo de lo que tipo de ítem se agrega, si el usuario desea puede poner la prioridad de esta que hacer, por ejemplo: Este trabajo, este trabajo, seria una alta prioridad, por el impacto que considero yo (como usuario).
\item[$\ast$ ]Hacer posible la edición de los ítems creados
\subitem$\ast$ En algún momento el usuario va a querer editar algo de lo que se creo. Como la fecha, prioridad, entre otras.
\item[$\ast$ ]Crear Grupos para organizar
\subitem$\ast$ La explicación de este atributo es sencillo, los argumentos son iguales al punto anterior (Página: \pageref{sec:quehaceres}).
\item[$\ast$ ]Incluir un cuadro de selección para marcar mostrar que se creo.
\subitem$\ast$ En una lista cómo sería sería util para poder marcar un ítem como finalizado y así que se borre del que hacer. Sin intervención adicional del usuario.
\item[$\ast$ ]Eliminación de un ítem creado.
\subitem$\ast$ Si el usuario desea eliminar un ítem que no finalizó y definitivamente no lo va a hacer lo va a querer borrar de la lista, no va a querer que ese ítem este ahí toda la vida.
\end{description}

% End of the document
\end{document}